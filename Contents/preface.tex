%%%%%%%%%%%%%%%%%%%%%%preface.tex%%%%%%%%%%%%%%%%%%%%%%%%%%%%%%%%%%%%%%%%%
% sample preface
%
% Use this file as a template for your own input.
%
%%%%%%%%%%%%%%%%%%%%%%%% Springer %%%%%%%%%%%%%%%%%%%%%%%%%%

\preface

%% Please write your preface here
These are notes for a course of lectures given at Princeton University during the academic year 1965-66. The subject of the lecture was compact Riemann surfaces, considered as complex analytic manifolds. There are already several expositions of this subject from basically the same point of view; the foremost is undoubtedly Hermann Weyl's ``Die Idee der Riemannschen Fl\"ache, (The Idea of the Riemann Surface)'' and most of the later treatments have followed Weyl's approach to a large degree. During recent years there has been considerable activity in the study of complex analytic manifolds of several dimensions, and various new tools and approaches have been developed. The aim of the lectures, in addition to treating of a beautiful subject for its own sake, was to introduce the students to some of these techniques in the case of one complex variable, where things are simpler and the results more complete.\par

The material covered is indicated by the table of contents. No familiarity with manifolds, sheaves, or sheaf cohomology was assumed, so those subjects are developed ab initio (from the beginning), although only so far as necessary for the purposes of the lectures. On the other hand, no attempt was made to discuss in detail the topology of surfaces; for that is really another subject, and there are very good treatments available elsewhere. The basic analytic tool used was the Serre duality theorem, rather than the theory of harmonic integrals or harmonic functions. The detailed treatment of the analytic properties of compact Riemann surfaces begins only in Chapter VII. Unfortunately, there was not enough time to get very far in the discussions; os the lectures have the air of being but an introduction to the subject. This may explain some of the surprising omissions, also. I hope to have an opportunity to continue the discussion further sometime.\par

With the possible exception of parts of Chapter IX, there is nothing really new here. References to the literature are scattered throughout, with no attempt at completeness. In addition to these and to the book of Hermann Weyl, the following general references should be mentioned here: Paul Appell and Edouard Goursat, ``Th\'eorie des Functions Alg\'ebriques,'' (gauthier-Villars, 1930); Kurt Hensel and Georg Landsberg, ``Theorie der algebraischen Funktionen einer Variablen,'' (Teubner, 1902; Chelsea, 1962); and Jean-Pierre, ``Corps locaux,'' (Hermann, 1962).\par

I should like to express my thanks here to Richard Hamilton, Henry Laufer, and Richard Mandelbaum for many suggestions and improvements; and to Elizabeth Epstein for typing the manuscript.

 

 \vspace{\baselineskip}
 \begin{flushright}\noindent
 Princeton, New Jersey,\hfill {\it R. C. Gunning}\\
 May, 1966 \hfill \\
 \end{flushright}


